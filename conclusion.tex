\chapter{Conclusion}

\paragraph{}
As part of the project we improved the existing english question answer system in Yioop and integrated a hindi question answer system in Yioop. The system uses a rule based approach for tagging and generating triplets for Hindi documents and storing them as part of the index. The system is able to handle simple questions asked by the user.

\paragraph{}
This project is one of the steps to improving and implementing a open source question answer module for two languages english and Hindi respectively. These modules are able to answer simple \textit{Wh} questions in both languages. We have performed some preliminary tests for these languages by creating separate indexes for english and Hindi. The systems were then evaluated by comparing the retrieved answers to a dataset of answers known to be true.

\paragraph{}
Our systems for English and Hindi are open domain systems. We used similar techniques to evaluate the systems by creating a set of questions and answers which are know to be true. We then evaluated our system by asking it questions from the question set and comparing retrieved answers with the known dataset. Our systems performance was observed to be comparable to the accuracy of open domain systems with accuracy for English at 63\% and for Hindi observed to be 55\%. Our system is outperformed by closed domain Hindi question answer systems IRSL 79\%, WebBased QA 
\cite{nanda2016hindi} 88\%. The factors which play a role are the type of approach used to perform the question answering. Using a rule based approach although faster has it limitation as implementing it requires knowledge about the language. Also, the type of information which is processed is far restricted for a  closed domain system as compared to open domain system.

\paragraph{}
We have followed a rule based approach in implementing the question answer system. Our systems can be improved going ahead by adding improving the text parsing for Hindi webpages. Also we can improve the the parsing of sentences extracted from web page summary by identifying the named entities using  the named entities added to the lexicon by the user. Other changes overtime may include modifying the rules used to implement the part of speech tagger to ensure that more information can be extracted from the text.