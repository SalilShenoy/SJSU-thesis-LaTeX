\chapter{Introduction}

\paragraph{}
The invention of computers in the nineteenth century meant that people could use technology to solve simple as well complex problems. Then came the Internet, which meant that devices could now be connected and used to share information. Finally, in 1989 the world was introduced to the World Wide Web, which is a set of global connected documents and resources. Today, we have a large reserves of data available world wide web which means information is only a click away. One of the biggest challenges we face is a efficient mechanism to process this data and make relevant and useful information available to the user. This is where Natural Language Processing \cite{chowdhury2003natural} has an important role to play.

\paragraph{}
Natural Language Processing is a discipline of Computer Science, Artificial Intelligence which helps humans interact with computers in a natural manner. We can define Natural Language Processing as an area of research which explore how machines can be programmed to understand and play with natural language text and speech wise. This involved studying of humans perceive and act to the language so that efficient tools and programs can be developed to manipulate a natural language. Natural Language processing can be used in Computer Science, Linguistics, Artificial Intelligence and Robotics. 

\paragraph{}
For the project we have implemented a system which has Natural Language Processing at its core. We discuss about improving an existing Question Answering Module in Yioop and then implement a similar system for Hindi which is a Indian language. We discuss the different approaches which are followed while developing a Question Answering System. The different tasks which are performed to process the data and convert it to a format which Yioop can use to handle simple English and Hindi questions. We also provide a way for the user to add entities to the Yioop database which can further improve the efficiency of the Question Answer System.
