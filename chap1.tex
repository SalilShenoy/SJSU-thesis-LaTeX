\chapter{Introduction}

\paragraph{}
The web has information written in almost 7000 languages and it is important to
have systems which can retrieve information effectively in different languages. The most researched Indian language is Hindi which means a relatively large number of Hindi documents are available. The purpose of implementing a question answer system which can accept and retrieve answers in Hindi is  to increase the access of Hindi speakers to more advanced information retrieval software. The users prefer and feel a sense of comfort if they can use a software, in a language they have expertise in. This is where Natural Language Processing \cite{chowdhury2003natural} plays an important role as it helps to handle information in native languages. I have implemented  a Hindi question answer system for Yioop which allows user to ask a question in Hindi and returns the answer also in Hindi. I developed a part of speech tagger and a triplet extractor to process Hindi text extracted by Yioop.

\paragraph{}
There are various Question Answering systems implemented over the years which extract data from the web and return answers. The systems implemented may use a knowledge store, machine learning or a combination of the two to process the data and extract question answers. We describe some of the related work and existing systems for Question Answering systems.

\paragraph{}
Question answer systems were developed with a view of extending research in natural language processing. One of the first question answer system developed was STUDENT. This system was capable of solving high school algebra problems. Another example,  was LUNAR which was a system developed to answer questions related to moon rock data. LUNAR answer questions with an accuracy of 78\%. An Example of a Hindi question answer system is IRSL which answered questions related to the Indian penal code. HLIDB is a knowledge based question answer system which takes input for user in Hindi and converts it to SQL and retrieves answers from the database. LADDER (Language Access to Distributed Data with Error Recovery) was designed for US Navy ships with a natural language interface. It takes queries in English language. The system was designed to help the navy decision makers. CHAT-80 was a question answer system implemented in prolog. It consists of facts about 150 countries of the world and a small set of English vocabulary which it uses to fetch relevant answers from the database. TEAM was designed to be easily configurable by database administrators with no knowledge of NLIDB. ASK allowed end users to teach the system new words during the interaction. ASK transparently generates suitable request to the appropriate underlying system for the user request. PRECISE was developed at the University of Washington. The database is in a relational database which uses SQL as the query language. It is called a Generic Interactive Natural Language Interface to Database. The user submits the query in English language. The query is analyzed both syntactically as well as semantically, so the question can be answered efficiently with respect to system knowledge base. Then the construction of valid SQL statement that represent the user's query is done and to retrieve the result of query to the user.

\paragraph{}
START, Question Answer System \cite{katz1997annotating} which extracts data from the web to answer questions ranging from cities to people etc. The system uses a database to fetch the answer to a user asked question. START system was tested using some sample questions related to various domain, and most of them were answered. Resluts were precise and few of them were supported with images. Garima Nanda et al. implemented a Hindi Question Answer system \cite{nanda2016Hindi} using a combination of machine learning and a knowledge base to answer user queries. The system parses input, tokenizes and extract features using previously calculated results. The system uses  Naive Bayes as a classifier combined with known data store to return answers in Hindi. IRSL (Information retrieval system for laws) \cite {sangeetha2017information} is a closed domain Question Answer system developed for helping users get answers related to the Indian penal code. The system uses OpenNLP to preprocess the input and a Q-Learning algorithm to learn from user inputs and a Wordnet developed at Princeton Univresity to extract synonyms. In Google \cite {alupului2016question}, previously when the user was presented with links and references. Today, the users is presented with a short paragraph at the top of the search results. The system is capable of semantically distinguishing the questions. The system however does not provide an answer for all queries. 

\paragraph{}
The Question Answer Systems are mainly classified as Open Domain and Closed Domain Systems. 
Open domain systems are responsible for handling large amounts of data and wide range of questions. They are designed to answer questions about everything. Google, Wikipedia, etc. are examples of such systems. They depend on their knowledge store to tackle the questions and return the best answers.
Closed domain systems are designed to handle questions in a specific domain. For example, IRSL system is designed specifically to handle question-answers related to the Indian Judiciary. Such systems have a fixed set of documents which they process when a user asks a query to return the best answer.

\paragraph{}
The current work on Hindi information retrieval systems is limited to closed domain systems. These systems cannot be used by people from outside that domain. I have added an open domain Hindi Question Answer system to Yioop which answers simple questions asked in Hindi by retrieving answers also in Hindi, from the internet. 

\paragraph{}
Natural Language Processing is the core of any Question Answer system. The number of modules involved in building an efficient Question Answer system adds to the complexity of such a system. As part of the project, we have improved the performance of an existing Question Answering Module \cite {patel2015question} in Yioop. The work includes implementing a similar Question Answer system for Hindi which is an Indian language. In this report, we describe the the different steps we followed to build the system. The report is organized as follows: Chapter 2 gives a background on Question Answer Systems and describes the different approaches used in developing such systems, Chapter 3, describes the individual modules  like part of speech tagger, triplet extractor which were implemented to build the Hindi Question Answer System, and also a feature which allows users to add named entities to the Yioop database, Chapter 4, describes tests and experiments conducted on the Hindi Question Answer System, and a comparison of English and Hindi Question Answer systems in Yioop, Chapter 5 we give a short summary of the work done.

